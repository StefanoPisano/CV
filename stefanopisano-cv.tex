%----------------------------------------------------------------------------------------
%	DOCUMENT DEFINITION
%----------------------------------------------------------------------------------------

% we use article class because we want to fully customize the page
\documentclass[10pt,A4]{article}	


%----------------------------------------------------------------------------------------
%	ENCODING
%----------------------------------------------------------------------------------------

%we use utf8 since we want to build from any machine
\usepackage[utf8]{inputenc}	

%----------------------------------------------------------------------------------------
%	LOGIC
%----------------------------------------------------------------------------------------

\usepackage{xifthen}
\usepackage{calc}
\usepackage[hidelinks]{hyperref}

%----------------------------------------------------------------------------------------
%	FONT
%----------------------------------------------------------------------------------------

% some tex-live fonts - choose your own

%\usepackage[defaultsans]{droidsans}
%\usepackage[default]{comfortaa}
%\usepackage{cmbright}
\usepackage[default]{raleway}
%\usepackage{fetamont}
%\usepackage[default]{gillius}
%\usepackage[light,math]{iwona}
%\usepackage[thin]{roboto} 

% set font default
\renewcommand*\familydefault{\sfdefault} 	
\usepackage[T1]{fontenc}

% more font size definitions
\usepackage{moresize}		

% font icons package
\usepackage{fontawesome}

%----------------------------------------------------------------------------------------
%	PAGE LAYOUT  DEFINITIONS
%----------------------------------------------------------------------------------------

%define page styles using geometry
\usepackage[a4paper]{geometry}		

% for example, change the margins to 2 inches all round
\geometry{top=2cm, bottom=2cm, left=2cm, right=2cm} 	

% use customized header
\usepackage{fancyhdr}				
\pagestyle{fancy}

%less space between header and content
\setlength{\headheight}{-5pt}		

% customize header entries
\lhead{}
\rhead{}
\chead{}

%indentation is zero
\setlength{\parindent}{0mm}

%----------------------------------------------------------------------------------------
%	GRAPHICS DEFINITIONS
%---------------------------------------------------------------------------------------- 

% for drawing graphics and charts
\usepackage{tikz}
\usetikzlibrary{shapes, backgrounds}

% use to vertically center content
% credits to: http://tex.stackexchange.com/questions/7219/how-to-vertically-center-two-images-next-to-each-other
\newcommand{\vcenteredinclude}[1]{\begingroup
	\setbox0=\hbox{\includegraphics{#1}}%
	\parbox{\wd0}{\box0}\endgroup}

% use to vertically center content
% credits to: http://tex.stackexchange.com/questions/7219/how-to-vertically-center-two-images-next-to-each-other
\newcommand*{\vcenteredhbox}[1]{\begingroup
	\setbox0=\hbox{#1}\parbox{\wd0}{\box0}\endgroup}

%----------------------------------------------------------------------------------------
%	ICON-SET EMBEDDING
%---------------------------------------------------------------------------------------- 

% at this point we simplify our icon-embedding by simply referring to a set of png images.
% if you find a good way of including svg without conflicting with other packages you can
% replace this part
\newcommand{\icon}[2]{\colorbox{black}{\makebox(#2, #2){\textcolor{white}{\large\csname fa#1\endcsname}}}}	%icon shortcut
\newcommand{\icontext}[3]{ 						%icon with text shortcut
	\vcenteredhbox{\icon{#1}{#2}}\hspace{0.2cm}\vcenteredhbox{\textcolor{black}{#3}}
}

%----------------------------------------------------------------------------------------
% 	HEADER
%----------------------------------------------------------------------------------------

% remove top header line
\renewcommand{\headrulewidth}{0pt} 

%remove botttom header line
\renewcommand{\footrulewidth}{0pt}	  	

%remove pagenum
\renewcommand{\thepage}{}	

%remove section num		
\renewcommand{\thesection}{}			


%----------------------------------------------------------------------------------------
%
% 	TIKZ GRAPHICS
%
%----------------------------------------------------------------------------------------

\newcounter{a}
\newcounter{b}
\newcounter{c}
\newcounter{barcount}

%----------------------------------------------------------------------------------------
% 	BAR CHART
%----------------------------------------------------------------------------------------

% draw a bar chart
% param 1: width
% param 2: height
% param 3: border color
% param 4: label text color
% param 5: label bg color
% param 6: cat 1 color
\newenvironment{barchart}[8]{
	
	\newcommand{\barwidth}{0.35}
	\newcommand{\barsep}{0.2}
	
	% param 1: overall percent
	% param 2: label
	% param 3: cat 1 percent
	% param 4: cat 2 percent
	% param 5: cat 3 percent
	\newcommand{\baritem}[5]{
		
		\pgfmathparse{##3+##4+##5}
		\let\perc\pgfmathresult
		
		\pgfmathparse{#2}
		\let\barsize\pgfmathresult
		
		\pgfmathparse{\barsize*##3/100}
		\let\barone\pgfmathresult
		
		\pgfmathparse{\barsize*##4/100}
		\let\bartwo\pgfmathresult
		
		\pgfmathparse{\barsize*##5/100}
		\let\barthree\pgfmathresult
		
		\pgfmathparse{(\barwidth*\thebarcount)+(\barsep*\thebarcount)}
		\let\barx\pgfmathresult
		
		\filldraw[fill=#6, draw=none] (0,-\barx) rectangle (\barone,-\barx-\barwidth);
		\filldraw[fill=#7, draw=none] (\barone, -\barx) rectangle (\barone+\bartwo,-\barx-\barwidth);
		\filldraw[fill=#8, draw=none] (\barone+\bartwo,-\barx ) rectangle (\barone+\bartwo+\barthree,-\barx-\barwidth);
		
		\node [label=180:\colorbox{#5}{\textcolor{#4}{##2}}] at (0,-\barx-0.175) {};
		\addtocounter{barcount}{1}
	}
	\begin{tikzpicture}
	\setcounter{barcount}{0}
	
}
{\end{tikzpicture}}

%----------------------------------------------------------------------------------------
% 	BUBBLE CHART
%----------------------------------------------------------------------------------------
\newcommand{\bubble}[5]{
	\definecolor{tmpcol}{RGB}{50,50,#5}
	% slice
	\filldraw[fill=black,draw=none] (#1,0.5) circle (#3);
	
	% outer label
	\node[label=\textcolor{black}{#4}] at (#1,0.7) {};
}

\newcommand{\bubbles}[2]{
	%reset counters
	\setcounter{a}{0}
	\setcounter{c}{150}
	\begin{tikzpicture}[scale=4]
	\foreach \p/\t in {#1} {
		\addtocounter{a}{1}
		\bubble{\thea/2}{\theb}{\p/25}{\t}{1\p0}
	}
	\end{tikzpicture}
}


%----------------------------------------------------------------------------------------
%	custom sections
%----------------------------------------------------------------------------------------

% create a coloured box with arrow and title as cv section headline
% param 1: section title
%
\newcommand{\cvsection}[1] {
	\textcolor{white}{\MakeUppercase{\textbf{#1}}}
}

\newcommand{\cvsect}[1]{
	\colorbox{black}{{\cvsection{#1}}}\\\\%
}

%----------------------------------------------------------------------------------------
% CUSTOM LOREM IPSUM
%----------------------------------------------------------------------------------------
\newcommand{\lorem}{Lorem ipsum dolor sit amet, consectetur adipiscing elit. Donec a diam lectus.}

%----------------------------------------------------------------------------------------
% ENTRY LIST
%----------------------------------------------------------------------------------------
\usepackage{tabularx}

\setlength{\tabcolsep}{0pt}
\newenvironment{entrylist}{%
	\begin{tabular*}{\textwidth}[t]{@{\extracolsep{\fill}}ll}
	}{%
	\end{tabular*}
}

\newcommand{\entry}[4]{%
	\parbox[t]{3.5cm}{%
		#1%
	}%
	&\parbox[t]{14cm}{%
		\textbf{#2}%
		\hfill%
		{\footnotesize \textbf{\textcolor{black}{#3}}}\\%
		#4%
	}\\\\}

\newcommand{\slashsep}{
	\hspace{2mm}/\hspace{2mm}
}

%----------------------------------------------------------------------------------------
%	DOCUMENT CONTENT
%----------------------------------------------------------------------------------------
\begin{document}
	
	%----------------------------------------------------------------------------------------
	%	TITLE HEADLINE
	%----------------------------------------------------------------------------------------
	\begin{minipage}[t]{0.60\textwidth}\hrule height 0pt width 0pt%
		\colorbox{black}{{\HUGE\textcolor{white}{\textbf{\MakeUppercase{Stefano
		 Pisano}}}}}%
		
		\vspace{1mm}\LARGE{Full Stack Developer}
	\end{minipage}%
	\begin{minipage}[t]{0.4\textwidth}\hrule height 0pt width 0pt%
		\small%
		\icontext{MapMarker}{12}{Milan, Italy}\\
		\icontext{Github}{12}{\href{https://github.com/StefanoPisano}{github.com/StefanoPisano}}\\
		\icontext{Linkedin}{12}{\href{https://www.linkedin.com/in/stefano-pisano/}{linkedin.com/in/stefano-pisano}}\\
		\icontext{At}{12}{\href{mailto:spisano@protonmail.com}{spisano@protonmail.com}}\\	
	
	\end{minipage}%

	
	% manage space by reducing font size
	\small%
	\vspace{0.5cm}

	%----------------------------------------------------------------------------------------
	%	SKILLS AND TECHNOLOGIES
	%----------------------------------------------------------------------------------------
	
	\cvsect{Who Am I?}%
	\begin{minipage}[t]{0.4\textwidth}%
		 I am an italian software developer based in Milan. I strongly believe in good practices and team working.I thrive in an environment where teamwork is appreciated and encouraged. \\
While writing code I always try to apply some magic tricks which will transform bad code into clean code.\\
My passion lies in keeping things simple. 
	\end{minipage}%
	\hfill
	\begin{minipage}[t]{0.5\textwidth}\hrule height 0pt width 0pt%
		\vspace{-10pt}%
		\begin{barchart}{10}{5.5}{red}{white}{black}{black}{black}{black}
			\baritem{50}{Java}{0}{0}{90}
			\baritem{50}{Spring}{0}{0}{80}
			\baritem{50}{SQL}{0}{0}{70}
			\baritem{50}{JavaScript}{0}{0}{60}
			\baritem{80}{Vue.js}{0}{0}{50}
			\baritem{80}{React}{0}{0}{20}
			\baritem{80}{AWS}{0}{0}{30}
			\baritem{40}{Git}{0}{0}{70}
		\end{barchart}
	\end{minipage}%
	
	\begin{center}
			\bubbles{4/Linux, 6/IntelliJ, 4/TDD, 3/Docker, 4/Team Leading}{}
	\end{center}
	
	\cvsect{Experience}
	\begin{entrylist}
		\entry
		{2020 – Present\\\footnotesize{Hybrid}}
		{Team Leader / Senior Full stack developer}
		{Tech Gap Italia}
		{As a Team Leader and Developer for an international project deployed on AWS, I coordinate the development team, ensure adherence to requirements, and craft effective solutions.\\
			\texttt{Java}\slashsep\texttt{AWS}\slashsep\texttt{SQL}\slashsep\texttt{Robot Framework}\slashsep\texttt{Vue.js}}
		\entry
		{2017 – 2020}
		{Full stack developer}
		{Anoki IT}
		{As a Full-Stack Developer on a project for a leading client in the game translation/localization industry, my responsibilities included overseeing the maintenance team and coordinating resources to resolve any client-reported issues. I contributed to code refactoring sessions and introduced new technologies such as Vue JS. To ensure high-quality code, we utilized SonarQube for static analysis. I also had the opportunity to serve as a tutor for new resources from time to time.\\
			\texttt{Java}\slashsep\texttt{SQL Server}\slashsep\texttt{JS}\slashsep\texttt{jQuery}\slashsep\texttt{Vue.js}\slashsep\texttt{JasperReports}}
		\entry
		{2015 – 2015\\\footnotesize{internship}}
		{Frontend Developer}
		{Università degli Studi dell'Insubria}
		{Implementation of a responsive layout for the website Arte Lab. However, the project presented a unique challenge, as the owner requested that I avoid using any frameworks such as Bootstrap. Despite this constraint, I was able to achieve my goal by leveraging media queries to create a fully responsive design.\\
			\texttt{JS}\slashsep\texttt{CSS}\slashsep\texttt{HTML}}
	\end{entrylist}
	\cvsect{Education}
	\begin{entrylist}
		\entry
		{2011 – 2016}
		{Bachelor's Degree in Computer Science}
		{Università degli Studi dell'Insubria (Varese, Italy)}
	
	\end{entrylist}
	\\
	\begin{minipage}[t]{0.3\textwidth}\hrule height 0pt width 0pt%
		\cvsect{Languages}
		\textbf{Italian} - native\\
		\textbf{English} - proficent
	\end{minipage}%
	\hspace{0cm}
	\begin{minipage}[t]{0.8\textwidth}\hrule height 0pt width 0pt%
		\cvsect{Hobbies}
		I like being a self-taught guitar player and photographer .I love travelling and hiking through the mountains and spending my time in the nature.
	\end{minipage}%
	\hspace{2cm}
	\\
	\\
	\\
	\begin{footnotesize}
			In compliance with the Italian legislative Decree no. 196 dated 30/06/2003, I hereby authorize you to use and process my personal details contained in this document.
	\end{footnotesize}


\end{document}
